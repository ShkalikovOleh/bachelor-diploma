%!TEX root = ../abstract.tex

\abstractUkr

Кваліфікаційна робота містить: 55 стор., 11 рисунки, 5 таблиць, 17 джерел.

Застосування сучасних нейромережевих архітектур у
задачі семантичної сегментації супутникових знімків
дозволяє має великий прикладний сенс. Проте незбалансованість
класів у навчальних вибірках, що є притаманним
саме супутниковим даним, є істотною проблемою, яка
знижує їх ефективність.

Дана робота полягає у розробці методів
застосуванні доповнення навчальних вибірок, що має
на меті підвищити якість семантичної сегментації.
Для цього було проаналізовано різні підходи до генеративних
моделей, зокрема генеративно-змагальні мережі.
Особлива увага була приділена архітектурі Pix2Pix
та її вдосконаленням. Також було розроблено і імплементовано
алгоритм генерації штучних масок класифікації і, як наслідок,
повний процес аугментації тренувального набору.

У ході виконання роботи було проведені експерименти, які
довели ефективність запропонованого підходу. Попри незначне
погіршення метрик для мажоритарних класів, якість класифікації
міноритарних класів значно зросла. Таким чином було отримано
метод аугментації наборів даних супутникових знімків,
який дозволяє покращити якість семантичної сегментації.

\MakeUppercase{СЕМАНТИЧНА СЕГМЕНТАЦІЯ,
    UNET,  НЕЗБАЛАНСОВАНІСТЬ КЛАСІВ,
    ГЕНЕРАТИВНО-ЗМАГАЛЬНІ МЕРЕЖІ, PIX2PIX,}

\abstractEng

The qualifying paper contains: 55 pages, 11 figures, 5 tables, 17 sources.

Using of modern neural network architectures in
the problem of semantic segmentation of satellite images
has a great practical meaning. However, the imbalance
of classes in the training samples, which is inherent in the
satellite data, is a significant problem that reduces
their effectiveness.

This work is to develop methods for applying the
augmentation of training datasets, which aims to improve
the quality of semantic segmentation. For this purpose,
different approaches to generative models have been analyzed,
in particular generative-adversarial networks. In particular,
the Pix2Pix architecture and its
improvements. An algorithm for generating artificial
classification masks and, as a result, a complete process
of augmentation of the training set has been also developed
and implemented.

In the course of the work, experiments have been conducted
that proved the effectiveness of the proposed approach.
Despite a slight deterioration in the metrics for majority
classes, the quality of the classification of minority
classes has increased significantly. Thus, a method of
augmentation of satellite image data sets has been obtained,
which allows to improve the quality of semantic
segmentation.

\MakeUppercase{SEMANTIC SEGMENTATION, UNET,
    CLASS IMBALANCE, GENERATIVE-ADVERSARIAL NETWORKS,
    PIX2PIX}

\clearpage