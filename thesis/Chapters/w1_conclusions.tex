%!TEX root = ../thesis.tex
% створюємо Висновки до всієї роботи
У результаті проведеної роботи була досліджена проблема
семантичної сегментації супутникових знімків, яка має важливу
роль у прикладних застосуваннях. Проте були виявлені та описані
проблеми, які виникають під час розв'язку цієї задачі сучасними
нейромережевими методами, а саме: складність створення великих
вибірок та сильна незбалансованість класів. Причому, на відміну від
інших сфер, отримати навчальні приклади, які могли б змінити розподіл
класів у багатьох випадках для супутникових даних не є можливим.

Для подолання цих проблем було розглянуто можливість аугментації
навчальної вибірки штучно згенерованими навчальними прикладами.
Для цього була дослідженні моделі, які дозволяють вирішити цю задачу,
а саме генеративно-змагальні мережі. При цьому, для вирішення проблеми
незбалансованості класів стандартні GAN не кращий вибір, бо
ми не можемо контролювати розподіл класів на згенерованих зображеннях.
Тому були опрацьовані моделі image-to-image translation, зокрема Pix2Pix,
які дозволяють генерувати супутникові знімки на основі масок.
У свою чергу модифікація вже існуючих масок, тобто зміна одних
класів на інші дозволяє скоригувати незбалансованість класів.

Таким чином було розроблено процес аугментації наборів даних супутникових
знімків, ефективність якого була перевірена
експериментальним чином (табл. \ref{tab:segm_result_augm_per_classes}, \ref{tab:segm_result_aug_global}).
Для генерації були використані різні модифікації
архітектури Pix2Pix, у тому числі та, яка
використовує дискримінатор та функції помилки з
архітектури Pix2PixHD, яка показала кращі результати, як з точки
зору метрик семантичної сегментації так і за якістю згенерованих
зображень.

Також було проведено дослідження різних функції помилки, які
застосовні при навчанні модифікації архітектури UNet, яка безпосередньо
і відповідає за семантичну сегментацію. Серед усіх досліджуваних підходів
найкращим виявився той, що використовував не зважену функцію похибки
Focal Loss. Його ж ми і застосовували при навчанні моделей
вже на доповнених вибірках.

У результаті ми отримали підтвердження того, що запропонований
підхід дозволяє значно підвищити якість семантичної сегментації
міноритарних класів, при цьому якість класифікації мажоритарних класів, хоч
у деяких випадках і стала нижчою, проте незначно.

Очевидні і подальші напрямки досліджень у даній сфері. По-перше, це
дослідження методів генерації, які б дозволили б врахувати той факт,
що супутникові знімки однієї території у різні пори року значно відрізняються.
По-друге, створення більш досконалих методів генерації масок.
І врешті решт, покращення існуючих генеративно-змагальних мереж
для генерації ще більш реалістичних штучних супутникових знімків.
