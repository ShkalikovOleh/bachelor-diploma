%!TEX root = ../thesis.tex
% створюємо вступ
\textbf{Актуальність дослідження.}
У сучасному світі задачі сегментації супутникових знімків мають
широке прикладне значення, і використовуються для
побудов мап сільськогосподарських культур \cite{kussul2017deep}, детектування незаконної
вирубки лісів \cite{sat_logging}, звалищ, пожеж, тощо. Проте сучасні методи
розв'язку цієї задачі здебільшого є алгоритмами навчання з учителем
та потребують розмічених навчальних вибірок. Створення даного матеріалу,
а саме ground truth масок для навчання потребує значних людських ресурсів, натомість
велика кількість людино-годин може не призвести до значного покращення
результатів сегментації. До того ж, навіть при використанні
такого підходу, точність класифікації міноритарних класів
є низькою, це пов'язано з тим, що загальна площа, яку вони займають
на супутникових знімках з вибірки мала.

Ідея роботи полягає у тому, щоб застосувати генеративні моделі
для створення штучних навчальних вибірок, тим самим збільшити й урізноманітнити
навчальні вибірки для моделей сегментації.

\textbf{Метою дослідження} є покращення точності семантичної сегментації супутникових знімків.
\textbf{Задача дослідження} полягає у дослідженні застосування
генеративно-змагальних нейронних мереж для аугментації навчальних вибірок.
Для розв'язання задачі необхідно вирішити такі \textbf{завдання}:

\begin{enumerate}
    \item оглянути сучасні методів розв'язання задачі
          семантичної сегментації;
    \item дослідити проблеми, які виникають при
          сегментації супутникових знімків;
    \item дослідити архітектур GAN у задачі
          image-to-image translation та перспективи
          їх застосування для аугментації навчальних вибірок супутникових
          знімків;
    \item провести експерименти з генерації
          аугментованих навчальних вибірок;
    \item проаналізувати вплив аугментації на якість семантичної сегментації.
\end{enumerate}

\textbf{Об'єктом дослідження} є засоби, які дозволяють збільшити
якість семантичної сегментації супутникових знімків.

\textbf{Предметом дослідження} є методи аугментації навчальних
вибірок за допомогою генеративно-змагальних мереж.

При розв'язанні поставлених завдань використовувались такі \emph{методи дослідження}:
методи математичної статистики, математичного аналізу та методи
побудови та навчання глибоких нейронних мереж.

\textbf{Наукова новизна} отриманих результатів полягає у
застосуванні генеративно-змагальних мереж для генерації
навчальних прикладів, які можуть збалансувати тренувальні
вибірки супутникових знімків і, як наслідок, покращити
якість їх семантичної сегментації.

\textbf{Практичне значення} результатів полягає у підвищенні якості
сегментації міноритарних класів на супутникових знімках.

\textbf{Апробація результатів та публікації.}
Частина даної роботи була представлена на
ESA Living Planet Symposium (23-27 травня 2022 р., м. Бонна, ФРН)
та XVIII Науково-практичній конференції
студентів, аспірантів
та молодих вчених <<Теоретичні i прикладні проблеми фізики,
математики та інформатики>> (15-16 червня 2022 р., м. Київ).