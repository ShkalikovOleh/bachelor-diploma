\thispagestyle{empty}
\linespread{1.1}

\begin{center}
    {\bfseries
        НАЦІОНАЛЬНИЙ ТЕХНІЧНИЙ УНІВЕРСИТЕТ УКРАЇНИ \par
        <<КИЇВСЬКИЙ ПОЛІТЕХНІЧНИЙ ІНСТИТУТ \par
        імені Ігоря СІКОРСЬКОГО>>\par
        НАВЧАЛЬНО-НАУКОВИЙ ФІЗИКО-ТЕХНІЧНИЙ ІНСТИТУТ\par
        Кафедра математичного моделювання та аналізу даних}
\end{center}
\par

\linespread{1.1}
Рівень вищої освіти --- перший (бакалаврський)

Спеціальність (освітня програма) --- 113~Прикладна математика,

ОПП <<Математичні методи моделювання, розпізнавання образів та безпеки даних>>

\vspace{10mm}
\begin{tabularx}{\textwidth}{XX}
     & ЗАТВЕРДЖУЮ                                                      \\[06pt]
     & Завідувач кафедри                                               \\[06pt]
     & \rule{2.5cm}{0.25pt} Н.М. Куссуль                            \\[06pt]
     & <<\rule{0.5cm}{0.25pt}>> \rule{2.5cm}{0.25pt} \YearOfDefence~р.
\end{tabularx}

\vspace{5mm}
\begin{center}
    {\bfseries ЗАВДАННЯ \par}
    {\bfseries на дипломну роботу \par}
\end{center}

%%%%%====================================
% !!! Не чіпайте наступні три команди!
%%%%%====================================
\frenchspacing
\doublespacing          % інтервал "1,5" між рядками, тепер навічно
\setfontsize{14}

Студент: \underline{\reportAuthor} \par

1. Тема роботи: <<\emph{\reportTitle}>>,

керівник: \underline{\supervisorRegalia ~\supervisorFio}, \par
затверджені наказом по університету \No \rule{0.5cm}{0.25pt} від <<\rule{0.5cm}{0.25pt}>> \rule{2.5cm}{0.25pt} \YearOfDefence~р.

2. Термін подання студентом роботи: <<\rule{0.5cm}{0.25pt}>> \rule{2.5cm}{0.25pt} \YearOfDefence~р.

3. Вихідні дані до роботи: \emph{(впишіть вихідні дані до роботи)}

4. Зміст роботи: \emph{(впишіть теми та задачі, які ви розкриваєте у роботі; можна робити це попунктно)}

5. Перелік ілюстративного матеріалу: презентація доповіді.

6. Дата видачі завдання: 30 вересня \YearOfBeginning~р.

% Якщо перша частина завдання вилізе за сторінку - приберіть команду \newpage
% Календарний план є продовженням завдання, а не окремою частиною
\newpage
\thispagestyle{empty}

\begin{center}
    Календарний план
\end{center}

\renewcommand{\arraystretch}{1.5}
\begin{table}[h!]
    \setfontsize{14pt}
    \centering
    \begin{tabularx}{\textwidth}{|>{\centering\arraybackslash\setlength\hsize{0.25\hsize}}X|>{\setlength\hsize{2\hsize}}X|>{\centering\arraybackslash\setlength\hsize{1\hsize}}X|>{\centering\arraybackslash\setlength\hsize{0.75\hsize}}X|}
        \hline \No\par з/п                              & Назва етапів виконання дипломної роботи & Термін виконання & Примітка \\
        \hline
        1                                               &
        Узгодження теми роботи із науковим керівником   &
        01.09.2021 - 30.09.2021                         &
        Виконано                                                                                                                \\
        \hline
        2                                               &
        Огляд сучасних методів розв'язання задачі
        семантичної сегментації                         &
        01.10.2021 - 11.11.2021                         &
        Виконано                                                                                                                \\
        \hline
        3                                               &
        Дослідження проблем, які виникають при
        сегментації супутникових знімків                &
        12.11.2021 - 17.02.2021                           &
        Виконано                                                                                                                \\
        \hline
        4                                               &
        Дослідження архітектур GAN у задачі
        image-to-image translation                      &
        18.12.2021 - 15.02.2022                         &
        Виконано                                                                                                                \\
        \hline
        5                                               &
        Проведення експериментів з генерації
        аугментованих навчальних вибірок                &
        05.02.2022 - 15.05.2022                         &
        Виконано                                                                                                                \\
        \hline
        6                                               &
        Аналіз впливу аугментації на якість сегментації &
        16.05.2022 - 31.05.2022                           &
        Виконано                                                                                                                \\
        \hline
    \end{tabularx}
\end{table}

\renewcommand{\arraystretch}{1}
\begin{tabularx}{\textwidth}{>{\setlength\hsize{1.5\hsize}}X >{\setlength\hsize{0.5\hsize}}X >{\setlength\hsize{1\hsize}}X}
    Студент  & \rule{2.5cm}{0.25pt} & \reportAuthorShort \\[06pt]
    Керівник & \rule{2.5cm}{0.25pt} & \supervisorFio     \\
\end{tabularx}

\newpage
